\documentclass[10pt]{beamer}

\usepackage{paratype}
\RequirePackage{cmap}
\RequirePackage[T2A]{fontenc}
\RequirePackage[utf8]{inputenc}

% Set up fonts
%\RequirePackage{paratype} % Arial-like sans serif font
%\RequirePackage{DejaVuSansMono} % Free widespread monospace font with cyrillic support
%\RequirePackage{nimbussans} % Helvetica clone
\RequirePackage[bold,light]{nimbusmono} % Courier clone
%\RequirePackage{tempora}  % "Times" font

\RequirePackage[english,russian]{babel}

\usepackage{amsmath,mathrsfs,amsfonts,amssymb, mathtools}
\usepackage{graphicx, epsfig}
\usepackage{subfig}
\usepackage{setspace}
\usepackage{multirow}
\captionsetup{labelformat=empty}
\usepackage{wrapfig}
\usepackage{array}
\usepackage{multirow}
\usepackage{listings}
\usepackage{color}

\definecolor{dkgreen}{rgb}{0,0.6,0}
\definecolor{gray}{rgb}{0.5,0.5,0.5}
\definecolor{mauve}{rgb}{0.58,0,0.82}

\lstset{frame=tb,
  language=Java,
  aboveskip=3mm,
  belowskip=3mm,
  showstringspaces=false,
  columns=flexible,
  basicstyle={\small\ttfamily},
  numbers=none,
  numberstyle=\tiny\color{gray},
  keywordstyle=\color{blue},
  commentstyle=\color{dkgreen},
  stringstyle=\color{mauve},
  breaklines=true,
  breakatwhitespace=true,
  tabsize=3
}

\usepackage{color, colortbl}
\definecolor{lightRed}{RGB}{240, 170, 150}
\definecolor{lightGreen}{RGB}{170, 230, 150}
\definecolor{lightYellow}{RGB}{240, 230, 180}

\usepackage{changepage}

\setbeamerfont{frametitle}{size=\normalsize,family=\sffamily}
\usetheme{Warsaw}%{Singapore}%{Warsaw}%{Warsaw}%{Darmstadt}
\usecolortheme{sidebartab}
\setbeamertemplate{footline}[author]
\expandafter\def\expandafter\insertshorttitle\expandafter{%
	\insertshorttitle\hfill%
	\insertframenumber\,/\,\inserttotalframenumber}

\definecolor{beamer@blendedblue}{RGB}{3,91,170}
\setbeamercolor{color1}{bg=beamer@blendedblue,fg=white}
% отключить клавиши навигации
\setbeamertemplate{navigation symbols}{}

\usepackage{appendixnumberbeamer}
\usepackage{booktabs}
\usepackage[scale=2]{ccicons}

\usepackage{pgfplots}
\usepgfplotslibrary{dateplot}

\usepackage{xspace}
\newcommand{\themename}{\textbf{\textsc{metropolis}}\xspace}

\usepackage{svg}
\usepackage{todonotes}

\AtBeginSection[]{}

\setbeamertemplate{section in toc}{%
  \alert{$\bullet$}~\inserttocsection}

\graphicspath{{images/}{images2/}{../images/}}

\usepackage{makecell}
\usepackage{changepage}

\title[]{Применение методов формальной верификации для анализа правил фильтрации сетевого трафика}
%\subtitle{Подназвание доклада}

\author{Дурнов Алексей Николаевич}

% Russian
\institute[МФТИ]{
    Московский физико-технический институт\\
    Физтех-школа радиотехники и компьютерных технологий\\
    Кафедра инфокоммуникационных систем и сетей
    \vspace{0.3cm} \\
    \textbf{Научный руководитель:} к.ф-м.н. Ефанов Николай Николаевич \\
    \vspace{0.3cm} \\
    \textbf{Консультант:} Ларин Дмитрий Викторович

    }

\date{Москва, 2026 г.}

\setbeamercolor{footline}{fg=black}

\setbeamerfont{footline}{series=\bfseries}

\begin{document}

\begin{frame}
    \titlepage
    \thispagestyle{empty}
\end{frame}

%% ==========================================================================
%% СЛАЙД 1: Актуальность проблемы
%% ==========================================================================
\begin{frame}{Актуальность проблемы}
    \begin{block}{Рост киберугроз}
        \begin{itemize}
            \item Ежегодный рост числа кибератак на 20--30\%
            \item Усложнение атак: APT, многовекторные атаки, атаки на уровне приложений
        \end{itemize}
    \end{block}

    \begin{block}{Средства защиты класса NGFW}
        \begin{itemize}
            \item Next-Generation Firewall становится одним из ключевых элементов сетевой безопасности
            \item В России активно развивается рынок отечественных решений: Kaspersky, Positive Technologies, UserGate, Континент, Ideco и др.
            \item Оставшиеся зарубежные решения: Check Point, Fortinet, Palo Alto Networks и др.
        \end{itemize}
    \end{block}

    \begin{block}{Цена ошибки}
        \begin{itemize}
            \item Ошибки конфигурации политики безопасности --- одна из главных причин инцидентов ИБ
            \item Средний ущерб от утечки данных в России - 11.5 млн рублей
        \end{itemize}
    \end{block}
\end{frame}

\begin{frame}{Цель и задачи}
    \begin{block}{Цель:}
        Разработка системы автоматического анализа и верификации правил фильтрации
        сетевого трафика для выявления ошибок конфигурации
    \end{block}

    \begin{block}{Задачи:}
        \begin{enumerate}
            \item Провести исследование существующих методов анализа правил фильтрации
            \item Разработать унифицированную вендоро-независимую модель представления правил различных форматов.
            Модель должна поддерживать отечественные решения
            \item Выбрать и адаптировать методы формальной верификации для анализа правил
            \item Реализовать алгоритмы обнаружения ошибок конфигурации в наборах правил
            \item Разработать прототип системы анализа правил фильтрации
            \item Провести тестирование на реальных конфигурациях
        \end{enumerate}
    \end{block}
\end{frame}

\begin{frame}{Правила фильтрации сетевого трафика}
    \begin{block}{}
        Политика сетевой безопасности в первую очередь определяется набором
        \alert{правил фильтрации} сетевого трафика.
    \end{block}

    \begin{block}{Основные классы правил фильтрации:}
        \begin{enumerate}
            \item \textbf{ACL} (Access Control List) -- Cisco, Juniper, Huawei и др.
            \begin{itemize}
                \item Анализ на уровне \alert{L3--L4} (protocol, src/dst IP, src/dst port)
                \item Небольшое количество дополнительных опций, простые сетевые объекты, наборы портов
                \item \textbf{Линейный поиск} правил (first-match)
            \end{itemize}
            \item \textbf{iptables} -- Linux-системы, MikroTik
            \begin{itemize}
                \item Анализ на уровне \alert{L3--L4}
                \item Дополнительные опции и простые объекты
                \item Линейный поиск, но с \textbf{механизмом цепочек} (chains)
            \end{itemize}
            \item \textbf{Политики безопасности NGFW} -- Kaspersky, Positive Technologies, Fortinet и др.
            \begin{itemize}
                \item Анализ на уровне \alert{L7} (приложения, сервисы, пользователи)
                \item Богатый набор объектов: зоны, приложения и др.
                \item Часто используется нелинейный поиск правил
            \end{itemize}
        \end{enumerate}
    \end{block}
\end{frame}

\begin{frame}{Структура правил фильтрации}
    \begin{itemize}
        \item Приоритет правила
        \item Действие - allow/deny/chain/return
        \item Условие совпадения:
        \begin{itemize}
            \item Сетевые объекты (wildcard-маски, диапазоны IP, Geo-IP, FQDN)
            \item Пользователи (группы, профили, идентификаторы)
            \item Зоны безопасности и тип правила для работы с зонами
            \item Сервисы (наборы портов и протоколов)
            \item Приложения (прикладной протокол, сервис или группа приложений)
        \end{itemize}
        \item Расписания
        \item Груповой профиль безопасности - объединяющий профиль движков безопасности (IDPS, AV, DNS Securiry)
        \item Дополнительные данные (число срабатываний, время последнего срабатывания, время первого срабатывания и тд)
    \end{itemize}
\end{frame}

\begin{frame}{Типы ошибок конфигурации правил и объектов}
    \begin{block}{Правил:}
        \begin{itemize}
            \item[\textcolor{red}{\textbf{!}}] \textbf{Shadowing} --- правило затенено, т.е. никогда не срабатывает
            \item[\textcolor{red}{\textbf{!}}] \textbf{Redundancy} --- избыточное правило
            \item[\textcolor{red}{\textbf{!}}] \textbf{Generalization} --- правило перекрывает более специфичное
            \item[\textcolor{red}{\textbf{!}}] \textbf{Correlation} --- частичное пересечение правил
            \item \textbf{Expired} --- правило c истекшим сроком действия
            \item \textbf{Disabled} --- правило отключено
            \item \textbf{Unused} --- правило не используется
        \end{itemize}
    \end{block}

    \begin{block}{Объектов:}
        \begin{itemize}
            \item \textbf{Unattached} --- объект не привязан к правилу
            \item \textbf{Duplicate} --- дублирующий объект
            \item \textbf{Unused within rule} --- объект не используется с правилом
        \end{itemize}
    \end{block}
\end{frame}

\begin{frame}{Сравнение методов анализа правил фильтрации}
    \begin{table}
    \centering
    \resizebox{1\textwidth}{!}{
    \renewcommand{\arraystretch}{1.15}
    \begin{tabular}{|p{3.2cm}|p{2.5cm}|p{2.5cm}|p{2.5cm}|}
        \hline
        \textbf{Критерий} & \textbf{Семантический анализ} & \textbf{Тестирование на трафике} & \textbf{Формальная верификация} \\
        \hline
        Точность & Все типы ошибок & Не все логические ошибки & Все типы ошибок \\
        \hline
        Интерпретируемость & Легко & Может быть сложно & Легко \\
        \hline
        Требует реального трафика/исполнения & Нет & Да & Нет \\
        \hline
        Сложность & $O(n^2)$ & Высокая (много тестов) & Обычно $O(n)$, в худшем случае $O(n^2)$ \\
        \hline
        Проверка реального поведения & Нет & Да & Нет \\
        \hline
    \end{tabular}
    }
    \end{table}
\end{frame}

\begin{frame}{Формальная верификация}
    \begin{block}{Header Space Analysis - метод моделирования сети}
        \begin{itemize}
            \item Каждый пакет представлен как точка в многомерном пространстве заголовков сетевого пакета: $\{0, 1\}^n$, где $n$ - число битов в заголовке
            \item Для удобства вводят сцециальный символ wildcard $x$ для обозначения любого бита
            \item На этом пространстве определены теоретико-множественные операции: пересечение, объединение, разность, дополнение и т.д.
            \item Правило фильтрации - это функция отображения в этом простанстве.
        \end{itemize}
    \end{block}

    Множество пакетов P - это множество точек в пространстве заголовков,
    или объединение выражений с символами '0', '1' и 'x'. Это множество также можно задать с помощью булевой формулы.

    \[
        P = 0xx10 \cup 10100 \;\;\Longleftrightarrow\;\; F(X) = (\neg x_1 \land x_4 \land \neg x_5) \lor (x_1 \land \neg x_2 \land x_3 \land \neg x_4 \land \neg x_5)
    \]
\end{frame}

\begin{frame}{Формальная верификация}
    \begin{minipage}{\textwidth}
        \begin{minipage}{0.5\textwidth}
            \begin{block}{Binary Decision Diagrams (BDD)}
                \begin{itemize}
                    \item Компактное представление булевых функций
                    \item Эффективные операции: пересечение, объединение, проверка эквивалентности
                \end{itemize}
            \end{block}
        \end{minipage}%
        \hfill
        \begin{minipage}{0.67\textwidth}
            \centering
            \includegraphics[width=\textwidth]{bdd_my_example.png}
        \end{minipage}
    \end{minipage}
\end{frame}

\begin{frame}{Архитектура системы}
    \begin{figure}
        \centering
        \includesvg[width=0.9\textwidth]{system_architecture.drawio.svg}
    \end{figure}
\end{frame}

\begin{frame}{Система анализа правил фильтрации}
    \centering
    \hspace*{-0.7cm}
    \includegraphics[width=0.95\paperwidth,height=\paperheight,keepaspectratio]{policy_analysis.jpg}
\end{frame}

\begin{frame}{Система анализа правил фильтрации}
    \centering
    \hspace*{-0.7cm}
    \includegraphics[width=0.95\paperwidth,height=\paperheight,keepaspectratio]{security_policy.jpg}
\end{frame}

\begin{frame}{Заключение}
    \begin{block}{Выполнено:}
        \begin{itemize}
            \item Проведён анализ существующих методов верификации правил фильтрации
            \item Разработана унифицированная модель представления правил
            \item Выбран и адаптирован метод на основе BDD для анализа конфликтов
            \item Реализованы алгоритмы обнаружения основных типов аномалий
            \item Разработан прототип системы анализа
        \end{itemize}
    \end{block}

    \begin{block}{Практическая значимость:}
        \begin{itemize}
            \item Снижение рисков мисконфигурации сетевых устройств
            \item Автоматизация аудита политик безопасности
            \item Поддержка мультивендорной инфраструктуры, в том числе и отечественных решений
        \end{itemize}
    \end{block}
\end{frame}

\begin{frame}{Направления дальнейших исследований}
    \begin{block}{}
        \begin{itemize}
            \item Оптимизация работы алгоритма для больших наборов правил (>100'000 правил)
            \item Автоматическая оптимизация наборов с помощью анализа достижимости на модели сети
            \item Интеграция с системами управления политиками (policy orchestration)
        \end{itemize}
    \end{block}
\end{frame}

\begin{frame}[plain]
    \begin{center}
        \vspace{2cm}
        {\LARGE Спасибо за внимание!}
        \vspace{1cm}
    \end{center}
\end{frame}

\appendix

\begin{frame}{BDD}
    \begin{minipage}{0.48\textwidth}
        \includegraphics[width=\textwidth]{BDD.png}
    \end{minipage}\hfill
    \begin{minipage}{0.48\textwidth}
        \includegraphics[width=\textwidth]{BDD_simple.svg.png}
    \end{minipage}
\end{frame}

\end{document}